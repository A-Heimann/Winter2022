\documentclass[aspectratio=169]{beamer}
\usepackage[utf8]{inputenc}     %accept different input encodings
\usepackage{ulem}               %package for underlining
\usepackage{hyperref}           %hypertext
\usepackage{fourier-otf}        %font utopia imported
\usepackage{amsmath}            %mathematical facilities in LaTeX
\usepackage{listings}           %for showing code in Latex, i.e. Python
\usepackage{longtable}          %allows tables to flow over pages
\usepackage{url}                %verbatim with url sensitive line
\usepackage{enumerate}          %custom labels for lists
\usepackage{cleveref}           %intelligent cross referencing


\usetheme{AnnArbor}            % Theme Darmstadt (AnnArbor)
\usecolortheme{beaver}        %color seahorse (beaver)
\setbeamertemplate{caption}[numbered] %to get numbered captures

% Bibliography
\usepackage[backend=biber,style=IEEE]{biblatex} %compiler for bibliography, style (IEEE)
\addbibresource{library.bib} %load in bibliography

\title[Template]{\emph{Scientific Writing, Documentation, and Presenting}}
\subtitle[]{Markdown, Latex, and Git(Hub)}
\author[Alexander Heimann]{Alexander Heimann}
\institute[]{NCKU - Department of Geomatics}
\date[\today]{\today}

\titlegraphic{\vspace{-1em}\includegraphics[height=2cm]{./img/logo.jpg}}

\begin{document}

\frame{\titlepage}      % create title page

\begin{frame}           % add toc
    \frametitle{Table of Contents}
    \tableofcontents
\end{frame}

\AtBeginSection[]       % Start each section with a highlighted toc
{
    \begin{frame}
        \frametitle{Table of Contents}
        \tableofcontents[currentsection]
    \end{frame}
}

\section{Introduction}

\begin{frame}
    \frametitle{About}
    This beamer presentation demonstrates how to use some basic functions in \LaTeX

    How to write a text \textbf{bold}, or \textit{italic}, or \underline{underline}, or how to put them \textbf{\textit{\underline{all together}}}.
    How to epmhasize text:
    \begin{block}{Something}
        Test
    \end{block}
    Or in a different color?\\
    \begin{alertblock}{With a list}
        \begin{itemize}
            \item Item 1
            \item Item 2
        \end{itemize}
    \end{alertblock}
\end{frame}

\section{Example}

\begin{frame}[allowframebreaks]         % allows the frame to brake and go over several pages
    \frametitle{Different list types}
    List items can be changed by adding [I], [i], [+]:
    \begin{itemize}
        \item unnumbered, standard
    \end{itemize}
    \begin{enumerate}
        \item numbered standard
    \end{enumerate}
    \begin{itemize}
        \item[$+$] unnumbered +
    \end{itemize}
    \begin{enumerate}
        \item[I] numbered capital roman letters
        \item[I] Do you know what is 4 in roman letters?
    \end{enumerate}
    \begin{enumerate}
        \item[i] numbered small roman letters
        \item[i] Do you know what is 4 in roman letters?
    \end{enumerate}
\end{frame}

\begin{frame}
    \frametitle{Nested Lists}
    % Customized bullets
    \begin{itemize}
        \item First item
        \begin{itemize}
            \item[\textbf{?}] My question.
            \item[\$\$] Want to make money?
            \item[$\int$] Mathematical list.
            \item[$\blacksquare$] Black square.
            \item[\textit{Remark}] My final remark.
            \begin{itemize}
                \item An item with an \textsc{equation}:
                \[ \sum_{n=1}^{\infty} \frac{1}{n^2}=\frac{\pi^2}{6}\]
            \end{itemize} 
            \item Continue in list
        \end{itemize}
        \item last item
    \end{itemize}
\end{frame}

\subsection{Comparison}
\begin{frame}
    \frametitle{Which reference is better?}

    When we use references and citations we can choose several packages:
    \begin{center}
    \begin{columns}[T]
        \begin{column}{0.5\textwidth}
            LaTeX Style\\
            \begin{equation}
                E = mc^2\label{eq:einstein}
            \end{equation}
            \vfill
            This equation (\ref{eq:einstein}) is from Einstein, \cite{Einstein1919}
        \end{column}
        \begin{column}{0.5\textwidth}
            Cleveref Style\\
            \begin{align}
                a^2 + b^2 = c^2 \label{eq:pythagoras}\\
                1 + 1 = 2 \label{eq:nonsense}\\
                1+2  = \binom{a}{b} \label{eq:3}
            \end{align}
            \vfill
            And here are the \crefrange{eq:pythagoras}{eq:3}         
        \end{column}
    \end{columns}
\end{center}
\end{frame}

\begin{frame}
    \frametitle{Images}

    \begin{figure}[t]
        \centering
        \includegraphics[width=0.25\textwidth]{./img/einstein.png}
        \caption{This is a caption}
        \label{fig:einstein}
    \end{figure}

    Now I am referencing to this figure \ref{fig:einstein} and to the equation \ref{eq:einstein}.

\end{frame}

\section{References}

\begin{frame}
    \frametitle{References}
    \printbibliography
\end{frame}

\end{document}